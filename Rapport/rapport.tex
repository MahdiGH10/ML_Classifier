%% @Author: Ahmad Ben Taleb
%  @Date:   2026
%% @Class:  PFE ITBS V3.

\documentclass[a4paper, oneside, 12pt, final]{extreport}
\usepackage{graphicx}

\parindent 0cm
\usepackage{makeidx}
\usepackage{tcolorbox}
\makeindex
%\usepackage[french]{babel}
\usepackage[lined,boxed,commentsnumbered, english, ruled,vlined,linesnumbered]{algorithm2e}
\usepackage{amsthm}
\newtheorem{theorem}{Theorem}[chapter]
\newtheorem{definition}{Definition}[chapter]
\newtheorem{exemple}{Example}[chapter]


%\usepackage[nottoc]{tocbibind}
%\addcontentsline{toc}{section}{References}

\providecommand{\keywords}[1]{\textbf{\textit{Mots clés---}} #1}
\providecommand{\keywordss}[1]{\textbf{\textit{Keywords---}} #1}

\usepackage{etoolbox}
%\makeatletter
%\patchcmd{\thebibliography}{%
%  \chapter*{\bibname}\@mkboth{\MakeUppercase\bibname}%{\MakeUppercase\bibname}}{%
%  \section{References}}{}{}
%\makeatother



\usepackage[nottoc]{tocbibind}

\textwidth 18cm
\textheight 24cm
\topmargin -0.5cm
\oddsidemargin -1cm

% set font encoding for PDFLaTeX or XeLaTeX
\usepackage{ifxetex}
\ifxetex
  \usepackage{fontspec}
\else
  \usepackage[T1]{fontenc}
  \usepackage[utf8]{inputenc}
  \usepackage{lmodern}
\fi


% Enable SageTeX to run SageMath code right inside this LaTeX file.
% documentation: http://mirrors.ctan.org/macros/latex/contrib/sagetex/sagetexpackage.pdf
%\usepackage{sagetex}


\newcommand{\reportTitle} {%
  %\textsc{Graduation Project Report}
  \textsc{Mini Projet Machine Learning}
}

\newcommand{\reportAuthor} {%
  Gharbi  \textsc{Mahdi}%
\\ \\ \\ \\
  Gharbi  \textsc{Skandar}%
}


  

\newcommand{\reportSubject} {%
    Articles Classification  Using
    \\ \\ Machine Learning%
}



\newcommand{\studyDepartment} {%
  Entreprise d'accueil 
  
}

\newcommand{\ITBS} {%
  %IT Business School
\\ Ecole Supérieure Privée des Technologies de l'Information et de Management de Nabeul
}

%\newcommand{\codePFE} {% Reference
%  Code PFE%
%}


\newcommand{\AU} {
\centering \textbf{Année Universitaire 2025-2026}
}


\newcommand{\specialcell}[1]{%
  \begin{tabularx}{\textwidth}{@{}X@{}}#1\end{tabularx}%
}

%%%%%%%%%%%%%%%%%%%%%%%%%%%%%%%%%%%%%%%%%%%%%%%%%%%%%%%
% Add your own commands here
%%%%%%%%%%%%%%%%%%%%%%%%%%%%%%%%%%%%%%%%%%%%%%%%%%%%%%%
\newcommand{\MyCommand} {%
  Does nothing really%
}


% used in maketitle
\title{\reportSubject}
\author{\reportAuthor}

% Enable SageTeX to run SageMath code right inside this LaTeX file.
% documentation: http://mirrors.ctan.org/macros/latex/contrib/sagetex/sagetexpackage.pdf
%\usepackage{sagetex}

%\hypersetup{
%  pdftitle={\reportTitle~-~\reportSubject},%
%  pdfauthor={\reportAuthor},%
%  pdfsubject={\reportSubject},%
%  pdfkeywords={report} {internship} {pfe} {enis}
%}

\usepackage{graphics}
\usepackage{graphicx}


\usepackage[acronym,toc,section=chapter]{glossaries}
\makeglossaries

\newacronym{abc}{ABC}{A contrived acronym}
\newacronym{efg}{EFG}{Another acronym}
\newacronym{svm}{SVM}{Support Vector Machines}

\pagenumbering{roman} 

\usepackage[utf8]{inputenc}
%\usepackage[french]{babel}

\begin{document}
\thispagestyle{empty}
\begin{titlepage}
\begin{center}


%%%%%%%%%%%%%%%%%%%%%%%%%%%%%%%%%%%%%%%%%%%%%%%
% THE HEADER
%%%%%%%%%%%%%%%%%%%%%%%%%%%%%%%%%%%%%%%%%%%%%%%

\includegraphics[scale=0.20]{embleme.jpg}
\vspace{0.5cm}

{%
  \fontsize{9pt}{9pt}\selectfont%
  \begin{tabular}{c}
    R\'epublique Tunisienne \\
    Minist\`ere de l'Enseignement Supérieur et de la Recherche Scientifique   \ITBS{}\\ 
  \end{tabular}
}

\vspace{1cm}

\includegraphics[scale=0.1]{logonoir.png}


%%%%%%%%%%%%%%%%%%%%%%%%%%%%%%%%%%%%%%%%%%%%%%%
% THE PAGE CONTENT
%%%%%%%%%%%%%%%%%%%%%%%%%%%%%%%%%%%%%%%%%%%%%%%

\vspace{10pt} {%
  \renewcommand*{\familydefault}{\defaultFont}
  \fontsize{46pt}{46pt}\selectfont%
  % MEMOIRE\\%
  %\reportTitle{}%\\\textsc{Report}\\%
}

\vspace{5pt}

\vspace{15pt}
{\textit{Rapport de Mini projet Machine learning}}\\

\vspace{10pt}
{\textbf{\large Diplôme National d'Ingénieur en Génie Informatique \\ Spécialité Génie Logiciel}}\\

%\includegraphics[scale=0.1]{logonoir.png}\\
\vspace{5pt}
\textbf{\textit{Réalisé par}}\\
\vspace{10pt} {%
  \fontsize{14pt}{14pt}\selectfont%
  {\bfseries\Large\sc \reportAuthor}\\
}%

\vspace{5pt} {%
  \renewcommand*{\familydefault}{\defaultFont}
  \fontsize{27pt}{27pt}\selectfont%
  \rule{0.5\textwidth}{.4pt}\\
  \vspace{7pt}
  \reportSubject{}\\%
  \vspace{7pt}
  \rule{0.5\textwidth}{.4pt}
}

\vspace{5pt}

%Soutenu le \dateSoutenance, devant la commission d'examen:\\
\vspace{10pt}

\begin{table}[h]
\begin{tabular}{lcr}
\textbf{Encadrant Académique:} M Ahmad Ben Taleb 
\end{tabular}
\end{table}




%\vfill

\vspace{40pt}



\centering
\includegraphics[scale=0.08]{logonoir.png}
\end{center}
\vspace{40pt}
\AU\\
\end{titlepage}

% ###############################
% # HELP COMMANDS               #
% ###############################
%
% -1 \part{part}
%  0 \chapter{chapter}
%  1 \section{section}
%  2 \subsection{subsection}
%  3 \subsubsection{subsubsection}
%  4 \paragraph{paragraph}
%  5 \subparagraph{subparagraph}


%%%%%%%%%%%%%%%%%%%%%%%%%%%%%%%%%%%%%%%%%%%%%%%%%%%%%%%
% Dédicace et Remerciements
%%%%%%%%%%%%%%%%%%%%%%%%%%%%%%%%%%%%%%%%%%%%%%%%%%%%%%%
% Nécessite : \usepackage{graphicx}, \usepackage{tcolorbox}, \usepackage{geometry} (pour marges A4, optionnel)
\thispagestyle{empty}
\noindent


\noindent



%\chapter*{D\'edicace}
%\addcontentsline{toc}{chapter}{Dedication}

%
%\nopagebreak{%
% And maybe a quote here
% \raggedright\hspace{5.75cm} To all of you,~\\
%\raggedright\hspace{7.75cm} I dedicate this work.
%  \raggedleft\normalfont\large\itshape{} \reportAuthor\par%
%}
%
%\cleardoublepage%


%\chapter*{Remerciements}
%\addcontentsline{toc}{chapter}{Thanks}







\begin{abstract}

This mini project aims to develop an automated news classification system using machine learning. We built a 4-class classifier for Business, Sports, Technology, and World news articles using TF-IDF features and supervised learning.

Working with 144,522 articles from AG News and 20 Newsgroups datasets, we compared three algorithms: Logistic Regression, LinearSVC, and Random Forest. The best model (LinearSVC, C=0.1) achieved 90.70\% accuracy and F1-score of 0.9068, demonstrating effective text classification performance. \\

\keywordss{Text Classification, Machine Learning, TF-IDF, Support Vector Machines, News Categorization} \\

\vspace{1.5cm}
\centerline{\bf Résumé}
\vspace{0.6cm}

Ce mini projet vise à développer un système automatisé de classification d'articles de presse utilisant l'apprentissage automatique. Nous avons construit un classificateur à 4 classes pour les articles d'Affaires, Sports, Technologie et Monde en utilisant TF-IDF et l'apprentissage supervisé.

Travaillant avec 144,522 articles des corpus AG News et 20 Newsgroups, nous avons comparé trois algorithmes : Régression Logistique, LinearSVC et Forêts Aléatoires. Le meilleur modèle (LinearSVC, C=0.1) a atteint 90.70\% de précision et un score F1 de 0.9068, démontrant une performance efficace en classification de texte. \\

\keywords{Classification de Texte, Apprentissage Automatique, TF-IDF, Machines à Vecteurs de Support, Catégorisation d'Articles}
\end{abstract}


%%%%%%%%%%%%%%%%%%%%%%%%%%%%%%%%%%%%%%%%%%%%%%%%%%%%%%%
% Divers chapitres
%%%%%%%%%%%%%%%%%%%%%%%%%%%%%%%%%%%%%%%%%%%%%%%%%%%%%%%
\addtocontents{toc}{\protect\setcounter{tocdepth}{4}}
\tableofcontents
%\addcontentsline{toc}{chapter}{\contentsname}

\listoffigures
%\addcontentsline{toc}{chapter}{Liste des Figures}
\listoftables
%\addcontentsline{toc}{chapter}{Liste des Tableaux}
\printglossaries

\cleardoublepage

\newpage
\pagenumbering{arabic}
\chapter*{Introduction}
\label{chap:general_intorduction}



\markboth{\MakeUppercase{Introduction}}{}%
\addcontentsline{toc}{chapter}{Introduction}%

\section*{Context and Motivation}

In today's digital age, news organizations and media platforms generate massive volumes of textual content daily. Automated document classification has become essential for organizing, retrieving, and recommending relevant information to users. Manual categorization is time-consuming, error-prone, and simply not scalable for modern content management systems.

This project addresses the challenge of \textbf{automatically classifying  articles into predefined categories} using machine learning techniques. Specifically, we focus on a 4-class classification problem covering: \textit{Business}, \textit{Sports}, \textit{Technology}, and \textit{World} news.

\section*{Problem Statement}

The core problem can be formulated as follows: Given a news article's textual content, predict which category it belongs to among four possible classes. This task presents several challenges:

\begin{itemize}
    \item \textbf{Semantic Ambiguity}: Articles may contain overlapping themes (e.g., technology in business)
    \item \textbf{High Dimensionality}: Text data requires transformation into numerical features
    \item \textbf{Generalization}: Models must perform well on unseen articles from diverse sources
    \item \textbf{Class Balance}: Ensuring fair representation across all categories
\end{itemize}

\section*{Objectives}

The main objectives of this work are:

\begin{enumerate}
    \item Collect and preprocess a large-scale, balanced dataset from multiple news sources
    \item Implement and compare multiple machine learning algorithms for text classification
    \item Apply appropriate feature engineering techniques (TF-IDF vectorization)
    \item Optimize model hyperparameters using cross-validation
    \item Achieve high classification performance (F1-score $\geq$ 0.85)
    \item Provide comprehensive evaluation with multiple metrics
\end{enumerate}

\section*{Approach and Contributions}

Our approach follows a rigorous machine learning pipeline: data collection from multiple sources (AG News, 20 Newsgroups), extensive preprocessing, feature extraction using TF-IDF with bigrams, training of three different algorithms (Logistic Regression, Support Vector Machines, Random Forest), hyperparameter tuning with 5-fold stratified cross-validation, and comprehensive evaluation.

The main contributions of this work include:
\begin{itemize}
    \item A well-balanced dataset of over 120,000 news articles across 4 categories
    \item Comparative analysis of three state-of-the-art classification algorithms
    \item Comprehensive evaluation using accuracy, precision, recall, F1-score, and ROC-AUC
    \item Production-ready models saved for deployment
\end{itemize}

\section*{Report Organization}

This report is organized as follows: 

Chapter \ref{chap:chapterone} presents the datasets used and the preprocessing pipeline, including data cleaning, text normalization, and exploratory data analysis. 

Chapter \ref{chap:2} describes the methodology, including feature engineering, model selection, hyperparameter tuning, and presents the experimental results with detailed performance comparisons.

Finally, the Conclusion synthesizes our findings and discusses future research directions. 





\chapter{Data studied}%
\label{chap:chapterone}


%%%%%%%%%%%%%%%%%%%%%%%%%%%%
% CHAPTER 1: DATASET       %
%%%%%%%%%%%%%%%%%%%%%%%%%%%%

This chapter presents the datasets used in this project, their characteristics, and the preprocessing pipeline applied to prepare the data for machine learning models.

\section{Data Sources}
\label{sec:datasources}

To ensure robust model training and generalization, we collected news articles from two well-established publicly available datasets:

\subsection{AG News Dataset}

The AG News Corpus is a collection of more than 1 million news articles gathered from over 2,000 news sources. For this project, we utilized a subset containing approximately 120,000 articles categorized into four classes. This dataset is widely used in natural language processing research and provides high-quality, professionally written news content.

\textbf{Dataset characteristics:}
\begin{itemize}
    \item \textbf{Source}: Academic benchmark dataset from AG News Corpus
    \item \textbf{Size}: ~120,000 articles
    \item \textbf{Categories}: World, Sports, Business, Sci/Tech
    \item \textbf{Format}: CSV with text and label columns
    \item \textbf{Language}: English
\end{itemize}

\subsection{20 Newsgroups Dataset}

The 20 Newsgroups dataset is a classic text classification benchmark containing approximately 18,000 newsgroup documents partitioned across 20 different newsgroups. We extracted and mapped relevant categories to align with our 4-class schema.

\textbf{Dataset characteristics:}
\begin{itemize}
    \item \textbf{Source}: Usenet newsgroup discussions
    \item \textbf{Size}: ~18,000 documents (subset used)
    \item \textbf{Original Categories}: 20 newsgroups
    \item \textbf{Mapped Categories}: Selected groups mapped to our 4 classes
    \item \textbf{Format}: Text files organized by category
\end{itemize}

% PLACEHOLDER FOR FIGURE
\begin{figure}[h]
    \centering
    \includegraphics[width=0.9\textwidth]{mergepic.png}
    \caption{Data collection from AG News and 20 Newsgroups datasets}
    \label{fig:datasources}
\end{figure}



\section{Category Mapping and Class Distribution}
\label{sec:classdist}

To create a unified 4-class classification problem, we performed careful mapping of subcategories from both datasets:

\begin{table}[h]
\centering
\begin{tabular}{|l|l|p{6cm}|}
\hline
\textbf{Class} & \textbf{Label} & \textbf{Description} \\ \hline
Business & 0 & Finance, economics, corporate news, stock markets, companies \\ \hline
Sports & 1 & Athletics, games, competitions, sports teams, tournaments \\ \hline
Tech & 2 & Technology, computing, AI, software, hardware, Internet \\ \hline
World & 3 & International news, politics, world events, diplomacy \\ \hline
\end{tabular}
\caption{Category definitions and label encoding}
\label{tab:categories}
\end{table}

After merging and mapping, we obtained a well-balanced dataset with the following distribution:

\begin{itemize}
    \item \textbf{Business}: ~30,000 articles
    \item \textbf{Sports}: ~30,000 articles
    \item \textbf{Tech}: ~30,000 articles
    \item \textbf{World}: ~38,000 articles
\end{itemize}

% PLACEHOLDER FOR FIGURE
\begin{figure}[h]
    \centering
    % INSERT: fig_class_distribution.png - Bar chart showing class counts
   \includegraphics[width=0.9\textwidth]{téléchargement.png}
    \caption{Distribution of articles across four categories}
    \label{fig:classdist}
\end{figure}

The relatively balanced distribution (variation within 25\%) eliminates the need for aggressive resampling techniques and allows the models to learn without strong class bias. That means the four classes have similar numbers of articles. 
The largest class (World: 38{,}000) is only about 27\% larger than the smallest classes (30{,}000). 
This is calculated as:

\[
\frac{38{,}000 - 30{,}000}{30{,}000} = 0.27 = 27\%.
\]


This eliminates the need for aggressive resampling techniques.

\begin{itemize}
    \item The model will not become biased toward predicting the largest class.
    \item All four categories receive roughly equal attention during training.
    \item The model learns fair patterns for each category.
\end{itemize}

\section{Data Preprocessing Pipeline}
\label{sec:preprocessing}

Text data requires extensive preprocessing to convert raw documents into a format suitable for machine learning algorithms. Our preprocessing pipeline consists of several sequential steps:

\subsection{Text Cleaning}

Raw text often contains noise that does not contribute to semantic understanding:

\begin{itemize}
    \item \textbf{HTML Removal}: Strip HTML tags and entities (e.g., \texttt{\&nbsp;}, \texttt{<br>})
    \item \textbf{Special Characters}: Remove non-alphanumeric characters except spaces.
    \item \textbf{URL Removal}: Eliminate URLs and email addresses.
    \item \textbf{Number Handling}: Replace or remove numeric values.
    \item \textbf{Extra Whitespace}: Normalize multiple spaces to single space.
\end{itemize}

\subsection{Text Normalization}

Normalization ensures consistency across the corpus:

\begin{itemize}
    \item \textbf{Lowercase Conversion}: Convert all text to lowercase to treat "Technology" and "technology" as the same token.
    \item \textbf{Tokenization}: Split text into individual words (tokens).
    \item \textbf{Stopword Removal}: Remove common words that carry little semantic meaning (e.g., "the", "is", "at", "which") using NLTK's English stopword list.
\end{itemize}

\subsection{Text Quality Assessment}

We performed quality checks to identify and handle problematic documents:

\begin{itemize}
    \item Documents with fewer than 10 words were flagged as potentially low-quality
    \item Duplicate articles were identified and removed
    \item Encoding errors (e.g., mojibake) were detected and corrected
\end{itemize}

% PLACEHOLDER FOR ALGORITHM
\begin{verbatim}
# Preprocessing function
def preprocess_text(text):
    # Remove HTML
    text = re.sub(r'<.*?>', '', text)
    # Lowercase
    text = text.lower()
    # Remove special characters
    text = re.sub(r'[^a-z0-9\s]', '', text)
    # Tokenize and remove stopwords
    tokens = [word for word in text.split() 
              if word not in stopwords]
    return ' '.join(tokens)
\end{verbatim}

\section{Exploratory Data Analysis}
\label{sec:eda}

Before model training, we conducted exploratory data analysis to understand the characteristics of our dataset.

\subsection{Text Length Distribution}

We analyzed the distribution of article lengths (word count) across categories:

% PLACEHOLDER FOR FIGURE
\begin{figure}[h]
    \centering
    % INSERT: fig_text_length_distribution.png - Box plots showing word count by category
    \includegraphics[width=1\textwidth, height=11cm]{mean.png}

    \caption{Distribution of article lengths (word count) by category}
    \label{fig:textlength}
\end{figure}

\textbf{Key findings:}
\begin{itemize}
    \item Average article length: 180-220 words
    \item Tech articles tend to be slightly longer (mean: 210 words)
    \item Sports articles are typically concise (mean: 185 words)
    \item Most articles fall between 100-300 words
\end{itemize}

\subsection{Vocabulary Analysis}

We examined vocabulary size and word frequency distributions:

\begin{itemize}
    \item \textbf{Total vocabulary}: ~75,000 unique tokens (before preprocessing)
    \item \textbf{After preprocessing}: ~50,000 unique tokens
    \item \textbf{Word frequency}: Follows Zipf's law (few very frequent words, many rare words)
\end{itemize}

% PLACEHOLDER FOR FIGURE
\begin{figure}[h]
    \centering
    % INSERT: fig_word_frequency.png - Log-log plot of word frequency
    \includegraphics[width=1\textwidth, height=10cm, ]{51.png}

    \caption{Word frequency distribution (log-log scale)}
    \label{fig:wordfreq}
\end{figure}

\subsection{Most Frequent Terms by Category}

Analyzing the most frequent terms in each category reveals discriminative features:

\textbf{Business}: company, market, stock, bank, financial, price, deal, CEO

\textbf{Sports}: game, team, player, win, season, coach, championship, score

\textbf{Tech}: software, computer, technology, Internet, data, system, digital, app

\textbf{World}: government, minister, country, president, international, security, political

% PLACEHOLDER FOR FIGURE
\begin{figure}[h]
    \centering
    % INSERT: fig_top_words_per_category.png - 4 bar charts showing top words
    \fbox{\parbox{0.7\textwidth}{\centering \textit{[PLACEHOLDER: Top 10 words per category]}\\ \textbf{fig\_top\_words\_category.png}}}
    \caption{Most frequent terms in each category}
    \label{fig:topwords}
\end{figure}

\section{Train-Validation-Test Split}
\label{sec:split}

To ensure unbiased evaluation, we split the dataset into three subsets:

\begin{table}[h]
\centering
\begin{tabular}{|l|r|r|}
\hline
\textbf{Subset} & \textbf{Size} & \textbf{Percentage} \\ \hline
Training Set & ~90,000 & 70\% \\ \hline
Validation Set & ~19,000 & 15\% \\ \hline
Test Set & ~19,000 & 15\% \\ \hline
\textbf{Total} & \textbf{~128,000} & \textbf{100\%} \\ \hline
\end{tabular}
\caption{Dataset split for training, validation, and testing}
\label{tab:datasplit}
\end{table}

\textbf{Stratification}: The split was performed using stratified sampling to maintain the same class distribution in all three subsets, ensuring that each set is representative of the overall data distribution.

\section{Summary}

This chapter presented the data foundation of our classification project. We combined two high-quality news datasets (AG News and 20 Newsgroups) to create a balanced corpus of ~128,000 articles across four categories. The preprocessing pipeline cleaned and normalized the text data, and exploratory analysis revealed distinct lexical patterns across categories. The dataset is now prepared for feature extraction and model training, which will be discussed in Chapter \ref{chap:2}.

\chapter{Models Used and Applications}
\label{chap:2}


%%%%%%%%%%%%%%%%%%%%%%%%%%%%
% CHAPTER 2: METHODOLOGY   %
%%%%%%%%%%%%%%%%%%%%%%%%%%%%

This chapter describes the methodology employed for building the news article classifier, including feature engineering, model selection, hyperparameter tuning, and evaluation metrics. We then present the experimental results and performance comparison.

\section{Feature Engineering: TF-IDF Vectorization}
\label{sec:tfidf}

Machine learning algorithms require numerical input, so text documents must be converted into numerical feature vectors. We employed \textbf{TF-IDF (Term Frequency-Inverse Document Frequency)} vectorization, a widely-used technique in text classification.

\subsection{TF-IDF Theory}

The TF-IDF score for a term $t$ in document $d$ from corpus $D$ is calculated as:

\begin{equation}
\text{TF-IDF}(t, d, D) = \text{TF}(t, d) \times \text{IDF}(t, D)
\end{equation}

where:
\begin{itemize}
    \item $\text{TF}(t, d)$ = frequency of term $t$ in document $d$
    \item $\text{IDF}(t, D) = \log\frac{|D|}{|\{d \in D : t \in d\}|}$
\end{itemize}

This approach assigns higher weights to terms that appear frequently in a document but rarely across the corpus, making them more discriminative.

\subsection{Implementation Parameters}

We configured the TF-IDF vectorizer with the following parameters:

\begin{itemize}
    \item \textbf{max\_features}: 10,000 (keep only the 10,000 most frequent terms)
    \item \textbf{ngram\_range}: (1, 2) (include both unigrams and bigrams)
    \item \textbf{min\_df}: 5 (ignore terms appearing in fewer than 5 documents)
    \item \textbf{max\_df}: 0.8 (ignore terms appearing in more than 80\% of documents)
    \item \textbf{sublinear\_tf}: True (apply logarithmic scaling to term frequency)
\end{itemize}

The inclusion of bigrams (2-word sequences) helps capture important phrases like "stock market" or "football team" that carry more semantic meaning than individual words.

\section{Model Selection and Training}
\label{sec:models}

We evaluated three supervised machine learning algorithms, each with different strengths for text classification.

\subsection{Logistic Regression}

Logistic Regression is a linear model that predicts class probabilities using the logistic function:

\begin{equation}
P(y=k|x) = \frac{e^{w_k^T x}}{\sum_{j=1}^{K} e^{w_j^T x}}
\end{equation}

\textbf{Advantages:}
\begin{itemize}
    \item Fast training and prediction
    \item Naturally handles multi-class classification (one-vs-rest or softmax)
    \item Provides probability estimates
    \item Interpretable coefficients
\end{itemize}

\textbf{Configuration:}
\begin{itemize}
    \item \textbf{Solver}: lbfgs (efficient for large datasets)
    \item \textbf{Multi-class}: multinomial
    \item \textbf{max\_iter}: 1000
\end{itemize}

\subsection{Support Vector Machine (LinearSVC)}

\gls{svm} finds the optimal hyperplane that maximally separates classes in feature space. For text classification, linear kernels are typically most effective.

The optimization objective is:

\begin{equation}
\min_{w,b} \frac{1}{2}\|w\|^2 + C\sum_{i=1}^{n}\max(0, 1-y_i(w^Tx_i + b))
\end{equation}

\textbf{Advantages:}
\begin{itemize}
    \item Effective in high-dimensional spaces (ideal for text with 10,000 features)
    \item Memory efficient (only support vectors matter)
    \item Robust to overfitting with proper regularization
\end{itemize}

\textbf{Configuration:}
\begin{itemize}
    \item \textbf{Loss}: squared\_hinge
    \item \textbf{C}: 0.1 (regularization parameter)
    \item \textbf{max\_iter}: 2000
\end{itemize}

\subsection{Random Forest}

Random Forest is an ensemble method that builds multiple decision trees and combines their predictions through majority voting.

\textbf{Advantages:}
\begin{itemize}
    \item Handles non-linear relationships
    \item Provides feature importance rankings
    \item Resistant to overfitting through ensemble averaging
    \item Requires minimal hyperparameter tuning
\end{itemize}

\textbf{Configuration:}
\begin{itemize}
    \item \textbf{n\_estimators}: 100 trees
    \item \textbf{max\_depth}: 50
    \item \textbf{min\_samples\_split}: 5
\end{itemize}

\section{Hyperparameter Tuning with Cross-Validation}
\label{sec:tuning}

To optimize model performance, we employed \textbf{GridSearchCV} with \textbf{5-fold stratified cross-validation}. This approach  searches through a predefined parameter grid and evaluates each combination using cross-validation.

\subsection{Cross-Validation Strategy}

Stratified K-Fold cross-validation divides the training data into $K=5$ folds while preserving class distribution in each fold. For each parameter combination:

\begin{enumerate}
    \item Train on 4 folds, validate on 1 fold
    \item Repeat 5 times (each fold serves as validation once)
    \item Average the 5 validation scores
\end{enumerate}

This provides a robust estimate of model performance and reduces overfitting to any single train-test split.
\begin{figure}[h]
    \centering
    % INSERT: fig_word_frequency.png - Log-log plot of word frequency
    \includegraphics[width=1\textwidth, height=12cm, ]{5fold.png}

    \caption{5 Fold Stratified Cross-Validation Process}
    \label{fig:wordfreq}
\end{figure}


\subsection{Parameter Grids}

\textbf{Logistic Regression:}
\begin{itemize}
    \item C: [0.1, 0.5, 1.0, 2.0]
    \item penalty: ['l2']
\end{itemize}

\textbf{LinearSVC:}
\begin{itemize}
    \item C: [0.1, 0.5, 1.0, 2.0]
    \item loss: ['hinge', 'squared\_hinge']
\end{itemize}

\textbf{Random Forest:}
\begin{itemize}
    \item n\_estimators: [100, 200]
    \item max\_depth: [30, 40, 50]
    \item min\_samples\_split: [2, 5]
\end{itemize}

\textbf{Total evaluations:} 40 model fits for LinearSVC (8 combinations $\times$ 5 folds)

\newpage
\begin{itemize}
    \item \textbf{Regularization (C)}: We tested values from 0.1 to 2.0 to find a good balance between keeping the model simple and letting it learn enough from the data.
    
    \item \textbf{SVM loss functions}: We used both \texttt{hinge} and \texttt{squared\_hinge} because \texttt{squared\_hinge} can handle noisy or similar news articles more smoothly.
    
    \item \textbf{Random Forest depth}: Depth values between 30 and 50 give the model enough flexibility without overfitting. The split values (2–5) work well for a large dataset.
    
    \item \textbf{Efficiency}: The search is still fast to run (for example, 40 training runs for LinearSVC with 5-fold cross-validation).
\end{itemize}
 



\section{Evaluation Metrics}
\label{sec:metrics}

We employed multiple evaluation metrics to comprehensively assess model performance:

\subsection{Classification Metrics}

\textbf{Accuracy:} Proportion of correct predictions
\begin{equation}
\text{Accuracy} = \frac{\text{Correct Predictions}}{\text{Total Predictions}}
\end{equation}

\textbf{Precision:} Proportion of true positives among predicted positives
\begin{equation}
\text{Precision}_k = \frac{TP_k}{TP_k + FP_k}
\end{equation}

\textbf{Recall:} Proportion of true positives among actual positives
\begin{equation}
\text{Recall}_k = \frac{TP_k}{TP_k + FN_k}
\end{equation}

\textbf{F1-Score:} Harmonic mean of precision and recall
\begin{equation}
F1_k = 2 \times \frac{\text{Precision}_k \times \text{Recall}_k}{\text{Precision}_k + \text{Recall}_k}
\end{equation}

For multi-class problems, we report both:
\begin{itemize}
    \item \textbf{Weighted F1}: Average weighted by class support
    \item \textbf{Macro F1}: Unweighted average across classes
\end{itemize}

\subsection{ROC-AUC Score}

The Receiver Operating Characteristic Area Under Curve (ROC-AUC) measures the model's ability to distinguish between classes. For multi-class classification, we use the one-vs-rest approach and average the AUC scores.

\section{Experimental Results}
\label{sec:results}

\subsection{Model Performance Comparison}

Table \ref{tab:modelcomp} presents the performance of all three models on the test set after hyperparameter tuning.

\begin{table}[h]
\centering
\begin{tabular}{|l|c|c|c|c|}
\hline
\textbf{Model} & \textbf{Accuracy} & \textbf{F1-Weighted} & \textbf{F1-Macro} & \textbf{ROC-AUC} \\ \hline
Logistic Regression & 0.9052 & 0.9050 & 0.9052 & 0.9845 \\ \hline
\textbf{LinearSVC (Best)} & \textbf{0.9070} & \textbf{0.9068} & \textbf{0.9070} & \textbf{0.9856} \\ \hline
Random Forest & 0.8815 & 0.8812 & 0.8815 & 0.9780 \\ \hline
\end{tabular}
\caption{Model performance comparison on test set}
\label{tab:modelcomp}
\end{table}

\textbf{Key findings:}
\begin{itemize}
    \item LinearSVC achieved the best overall performance with 90.70\% accuracy
    \item Logistic Regression performed nearly as well (90.52\%), with faster training time
    \item Random Forest showed slightly lower performance but provided interpretable feature importance
    \item All models exceeded the target F1-score of 0.85
\end{itemize}

% PLACEHOLDER FOR FIGURE
\begin{figure}[h]
    \centering
    % INSERT: fig_model_comparison.png - Bar chart comparing model metrics
    \includegraphics[width=1\textwidth, height=10cm,]{modelcomepratis.png}

    \caption{Comparison of model performance across metrics}
    \label{fig:modelcomp}
\end{figure}

\subsection{Best Model: LinearSVC}

The tuned LinearSVC model with parameters \texttt{C=0.1}, \texttt{loss='squared\_hinge'} was selected as the final model based on its superior performance.

\textbf{Per-class performance:}

\begin{table}[h]
\centering
\begin{tabular}{|l|c|c|c|c|}
\hline
\textbf{Class} & \textbf{Precision} & \textbf{Recall} & \textbf{F1-Score} & \textbf{Support} \\ \hline
Business & 0.91 & 0.89 & 0.90 & 4,500 \\ \hline
Sports & 0.95 & 0.97 & 0.96 & 4,500 \\ \hline
Tech & 0.88 & 0.87 & 0.87 & 4,500 \\ \hline
World & 0.90 & 0.91 & 0.90 & 5,700 \\ \hline
\end{tabular}
\caption{Per-class performance of best model (LinearSVC)}
\label{tab:perclass}
\end{table}

\textbf{Observations:}
\begin{itemize}
    \item Sports articles are easiest to classify (F1=0.96) due to distinct vocabulary
    \item Tech articles are most challenging (F1=0.87), likely due to overlap with Business
    \item All classes achieve F1-scores above 0.87, indicating balanced performance
\end{itemize}

\subsection{Confusion Matrix Analysis}

% PLACEHOLDER FOR FIGURE
\begin{figure}[h]
    \centering
    % INSERT: fig_confusion_matrix.png - Heatmap showing confusion matrix
   \includegraphics[width=0.9\textwidth, height=9cm]{faaa.png}

    \caption{Confusion matrix for LinearSVC on test set}
    \label{fig:confmat}
\end{figure}

The confusion matrix reveals:
\begin{itemize}
    \item Most errors occur between Business and Tech (expected due to tech companies)
    \item Very few Sports misclassifications (distinctive vocabulary)
    \item World and Business occasionally confused (political economy articles)
\end{itemize}

\subsection{ROC Curves}

% PLACEHOLDER FOR FIGURE
\begin{figure}[h]
    \centering
    % INSERT: fig_roc_curves.png - ROC curves comparing 3 models
    \includegraphics[width=1\textwidth, height=11cm]{aze.png}
    \caption{ROC curve of the winning model SVM}
    \label{fig:roc}
\end{figure}

The ROC curve for LinearSVC demonstrates excellent discrimination ability across all four news categories, with an AUC score exceeding 0.98. The curve's proximity to the top-left corner indicates high true positive rates with minimal false positives, significantly outperforming random classification (diagonal line).

\subsection{Feature Importance Analysis}

Using the model coefficients, we identified the most discriminative features for each class:

\textbf{Top predictive features:}

\begin{itemize}
    \item \textbf{Business}: "stock market", "company", "financial", "shares", "bank"
    \item \textbf{Sports}: "game", "team", "player", "win", "championship"
    \item \textbf{Tech}: "software", "technology", "computer", "internet", "digital"
    \item \textbf{World}: "government", "minister", "president", "security", "international"
\end{itemize}

% PLACEHOLDER FOR FIGURE
\begin{figure}[h]
    \centering
    % INSERT: fig_feature_importance.png - Top 10 features per class
    \includegraphics[width=1\textwidth, height=15.5cm]{dada.png}
    \caption{Top 10 most discriminative features per class}
    \label{fig:features}
\end{figure}

\subsection{Cross-Validation Scores}

The 5-fold cross-validation during hyperparameter tuning showed consistent performance:

\begin{itemize}
    \item \textbf{Mean CV F1-Score}: 0.9057
    \item \textbf{Standard Deviation}: 0.0023
    \item \textbf{Min-Max Range}: [0.9031, 0.9082]
\end{itemize}

The low variance indicates that the model generalizes well and is not sensitive to specific train-test splits.
\newpage
\section{Model Deployment}
\label{sec:deployment}

We serialized the following components using joblib:

\begin{itemize}
    \item Trained LinearSVC model (\texttt{best\_model.pkl})
    \item TF-IDF vectorizer (\texttt{tfidf\_vectorizer.pkl})
    \item Label encoder (\texttt{label\_encoder.pkl})
    \item Metadata (categories, parameters, performance metrics)
\end{itemize}

These artifacts can be loaded for real-time prediction on new articles.


\section{Discussion}
\label{sec:discussion}

\subsection{Why Linear Models Performed Best}

Linear models (Logistic Regression and LinearSVC) outperformed Random Forest for several reasons:

\begin{itemize}
    \item \textbf{High dimensionality}: With 10,000 TF-IDF features, the feature space is sparse and high-dimensional, where linear models excel
    \item \textbf{Text is inherently linear}: Document classification based on word presence/absence is often linearly separable
    \item \textbf{Overfitting prevention}: Simple linear decision boundaries generalize better than complex tree-based rules
\end{itemize}

\subsection{Impact of Bigrams}

Including bigrams (2-word phrases) alongside unigrams improved F1-score by approximately 2-3\%. Key examples:

\begin{itemize}
    \item "stock market" is more informative than "stock" + "market" separately
    \item "football team" clearly indicates Sports
    \item "software company" bridges Tech and Business
\end{itemize}

\subsection{Regularization Benefits}

The optimal regularization parameter \texttt{C=0.1} (strong regularization) prevented overfitting despite high dimensionality. Without regularization, validation performance dropped by 1-2\%.

\subsection{Limitations and Challenges}

\begin{itemize}
    \item \textbf{Business-Tech overlap}: Articles about tech companies often contain both business and technology terminology
    \item \textbf{Static categories}: Real-world news evolves (e.g., "crypto" wasn't common when these datasets were created)
    \item \textbf{Short articles}: Very brief articles (< 50 words) are harder to classify accurately
\end{itemize}

\section{Summary}

This chapter presented the complete methodology for building a 4-class news article classifier. Using TF-IDF feature extraction with bigrams, we trained and compared three machine learning algorithms. LinearSVC emerged as the best model, achieving 90.70\% accuracy and F1-score of 0.9068 through 5-fold cross-validation and hyperparameter tuning. The model demonstrates strong performance across all categories and is ready for production deployment.


\chapter*{Conclusion et Perspectives}
\label{chap:conclusion}
\markboth{\MakeUppercase{Conclusion et Perspectives}}{}%
\addcontentsline{toc}{chapter}{Conclusion et Perspectives}

\section*{Summary of Achievements}

This project successfully developed a production-ready machine learning system for automatic news article classification. We addressed the challenge of categorizing textual documents into four well-defined categories: Business, Sports, Technology, and World news.

Our main contributions include:

\begin{itemize}
    \item \textbf{Robust Dataset}: Compiled and preprocessed ~128,000 news articles from two established sources (AG News and 20 Newsgroups), ensuring balanced class distribution
    \item \textbf{Comprehensive Pipeline}: Implemented a complete text preprocessing pipeline including cleaning, normalization, and quality assessment
    \item \textbf{Feature Engineering}: Applied TF-IDF vectorization with bigrams, capturing both individual terms and meaningful phrases
    \item \textbf{Model Comparison}: Systematically evaluated three algorithms (Logistic Regression, LinearSVC, Random Forest) with rigorous cross-validation
    \item \textbf{Strong Performance}: Achieved 90.70\% accuracy with LinearSVC, exceeding the target F1-score of 0.85 across all classes
    \item \textbf{Deployment-Ready}: Serialized trained models and vectorizers for production use
\end{itemize}

\section*{Key Findings}

Several important insights emerged from this work:

\begin{enumerate}
    \item \textbf{Linear models excel for text classification}: Despite their simplicity, Logistic Regression and LinearSVC outperformed Random Forest in the high-dimensional TF-IDF feature space
    
    \item \textbf{Bigrams improve performance}: Including 2-word phrases alongside individual words increased F1-score by 2-3\%, capturing semantic units like "stock market" and "football team"
    
    \item \textbf{Regularization is crucial}: Strong regularization (C=0.1) prevented overfitting despite 10,000 features, demonstrating that simpler models generalize better
    
    \item \textbf{Class-specific challenges}: Sports articles were easiest to classify (F1=0.96) due to distinctive vocabulary, while Tech-Business overlap posed the greatest challenge (F1=0.87)
    
    \item \textbf{Cross-validation reliability}: Low variance in 5-fold CV scores (std=0.0023) confirmed that the model is robust and not dependent on specific data splits
\end{enumerate}

\section*{Practical Applications}

The developed classifier has immediate real-world applications:

\begin{itemize}
    \item \textbf{Content Management Systems}: Automatically tag and organize incoming news articles
    \item \textbf{Personalized Recommendations}: Route articles to users based on their interests
    \item \textbf{News Aggregation}: Filter and categorize content from multiple sources
    \item \textbf{Information Retrieval}: Improve search by category-specific indexing
    \item \textbf{Content Moderation}: Flag misclassified or ambiguous articles for human review
\end{itemize}

\section*{Limitations}

Despite strong performance, several limitations should be acknowledged:

\begin{itemize}
    \item \textbf{Category Overlap}: Articles about technology companies often blend Business and Tech themes, causing occasional misclassifications
    
    \item \textbf{Temporal Drift}: News vocabulary evolves (e.g., "cryptocurrency", "AI", "pandemic"), requiring periodic model retraining
    
    \item \textbf{Short Documents}: Articles with fewer than 50 words provide insufficient context for accurate classification
    
    \item \textbf{Domain Specificity}: The model is trained on general news and may not generalize well to specialized domains (medical, legal, scientific)
    
    \item \textbf{Multilingual Support}: Currently limited to English; extending to other languages requires new training data
\end{itemize}

\section*{Future Work and Perspectives}

Several promising directions for future research and development:

\subsection*{Deep Learning Approaches}

\begin{itemize}
    \item \textbf{Transformer Models}: Implement BERT, RoBERTa, or DistilBERT for contextualized word embeddings that capture semantic nuances better than TF-IDF
    
    \item \textbf{Transfer Learning}: Fine-tune pre-trained language models on news-specific corpora for improved performance with less training data
    
    \item \textbf{Multi-label Classification}: Extend the system to assign multiple categories when articles span several topics
\end{itemize}

\subsection*{Enhanced Features}

\begin{itemize}
    \item \textbf{Named Entity Recognition}: Extract and leverage entities (persons, organizations, locations) as additional features
    
    \item \textbf{Topic Modeling}: Incorporate latent topic distributions (LDA, NMF) as supplementary features
    
    \item \textbf{Sentiment Analysis}: Add sentiment scores to distinguish opinion pieces from factual reporting
    
    \item \textbf{Metadata Integration}: Include publication date, source, and author information
\end{itemize}

\subsection*{Model Improvements}

\begin{itemize}
    \item \textbf{Active Learning}: Implement uncertainty sampling to identify difficult cases for human annotation, improving model with minimal labeling effort
    
    \item \textbf{Online Learning}: Enable incremental model updates as new articles arrive, adapting to vocabulary drift
    
    \item \textbf{Ensemble Methods}: Combine predictions from multiple models (stacking) to improve robustness
    
    \item \textbf{Confidence Calibration}: Better calibrate probability outputs for risk-aware decision making
\end{itemize}

\subsection*{System Extensions}

\begin{itemize}
    \item \textbf{Real-time API}: Deploy as a REST API service with sub-second latency for production integration
    
    \item \textbf{Explainability Dashboard}: Develop a web interface showing which words/phrases influenced each prediction
    
    \item \textbf{A/B Testing Framework}: Continuously evaluate model variants in production
    
    \item \textbf{Multilingual Support}: Extend to French, Arabic, and other languages relevant for international news
\end{itemize}

\subsection*{Research Directions}

\begin{itemize}
    \item \textbf{Zero-shot Classification}: Investigate methods to classify articles into unseen categories without retraining
    
    \item \textbf{Few-shot Learning}: Enable rapid adaptation to new categories with minimal examples
    
    \item \textbf{Adversarial Robustness}: Test model resilience against intentionally misleading text or adversarial attacks
    
    \item \textbf{Bias Detection}: Analyze and mitigate potential biases in classification across different news sources
\end{itemize}

\section*{Concluding Remarks}

In this project, we showed that classic machine learning models can perform very well on text classification when they are used correctly. By cleaning the data, choosing good features, tuning the models with cross-validation, and carefully evaluating the results, we were able to build a classifier that reaches over 90\% accuracy.


The system is also easy to improve in the future. New features, different algorithms, or even deep-learning models can be added without changing the whole structure. The trained models are ready to be used in real applications, and the reported limitations give a clear idea of what can be improved next.

Overall, this work helps make news articles easier to organize and understand using automatic classification. This can be useful for both content creators and readers, especially today when information is growing faster than ever.
\newpage
\appendix
\addcontentsline{toc}{chapter}{Appendices}

\chapter{Python Code and Implementation}
\label{chap:appendix}


\section{Data Preprocessing Pipeline}

\begin{verbatim}
import re
import nltk
from nltk.corpus import stopwords

def clean_text(text):
    """Clean and preprocess text data"""
    # Lowercase
    text = text.lower()
    
    # Remove URLs
    text = re.sub(r'http\S+|www\S+', '', text)
    
    # Remove HTML tags
    text = re.sub(r'<.*?>', '', text)
    
    # Remove special characters and digits
    text = re.sub(r'[^a-zA-Z\s]', '', text)
    
    # Remove extra whitespace
    text = ' '.join(text.split())
    
    # Remove stopwords
    stop_words = set(stopwords.words('english'))
    words = text.split()
    text = ' '.join([w for w in words if w not in stop_words])
    
    return text

# Apply preprocessing
df['text_clean'] = df['text'].apply(clean_text)
\end{verbatim}


\section{TF-IDF Feature Extraction}

\begin{verbatim}
from sklearn.feature_extraction.text import TfidfVectorizer

# Initialize TF-IDF vectorizer with bigrams
vectorizer = TfidfVectorizer(
    max_features=10000,
    ngram_range=(1, 2),
    min_df=5,
    max_df=0.8,
    sublinear_tf=True
)

# Fit and transform training data
X_train = vectorizer.fit_transform(train_df['text_clean'])
X_test = vectorizer.transform(test_df['text_clean'])
\end{verbatim}


\section{Model Training}

\begin{verbatim}
from sklearn.svm import LinearSVC
from sklearn.model_selection import GridSearchCV, StratifiedKFold

# Define model
model = LinearSVC(max_iter=2000, random_state=42, dual=False)

# Hyperparameter grid
param_grid = {
    'C': [0.1, 0.5, 1.0, 2.0],
    'loss': ['hinge', 'squared_hinge']
}

# Grid search with cross-validation
grid_search = GridSearchCV(
    model,
    param_grid,
    cv=StratifiedKFold(n_splits=5, shuffle=True, random_state=42),
    scoring='f1_weighted',
    n_jobs=-1
)

# Train model
grid_search.fit(X_train, y_train)

# Best model
best_model = grid_search.best_estimator_
print(f"Best parameters: {grid_search.best_params_}")
\end{verbatim}


\section{Model Evaluation}

\begin{verbatim}
from sklearn.metrics import classification_report, confusion_matrix

# Predictions
y_pred = best_model.predict(X_test)

# Classification report
print(classification_report(y_test, y_pred, 
                          target_names=categories))

# Confusion matrix
cm = confusion_matrix(y_test, y_pred)
\end{verbatim}
\section{Libraries used}
 
  \begin{table}[h]
\centering
\begin{tabular}{|l|l|p{7cm}|}
\hline
\textbf{Library} & \textbf{Version} & \textbf{Purpose} \\ \hline
NumPy & 1.24+ & Numerical computing and array operations \\ \hline
Pandas & 2.0+ & Data manipulation and DataFrame handling \\ \hline
Scikit-learn & 1.3+ & Machine learning algorithms, TF-IDF, cross-validation, metrics \\ \hline
Matplotlib & 3.7+ & Data visualization and plotting \\ \hline
Seaborn & 0.12+ & Statistical visualizations and heatmaps \\ \hline
NLTK & 3.8+ & Natural language preprocessing (tokenization, stopwords) \\ \hline
Joblib & 1.3+ & Model serialization and persistence \\ \hline
\end{tabular}
\caption{Python libraries used in the project}
\label{tab:libraries}
\end{table}


%%%%%%%%%%%%%%%%%%%%%%%%%%%%%%%%%%%%%%%%%%%%%%%%%%%%
% Don't touch this, it is auto generated
%%%%%%%%%%%%%%%%%%%%%%%%%%%%%%%%%%%%%%%%%%%%%%%%%%%%
\nocite{*}

%\phantomsection{}
%\addcontentsline{toc}{chapter}{Webography}
%\printbibliography[title={Webography},type=online]

%\phantomsection{}
%\addcontentsline{toc}{chapter}{Bibliography}
%\printbibliography[title={Bibliography},nottype=online]

%\printbibheading %exemple de bibliographie divisée en sections. Pour ajouter des oeuvres non citées,utiliser \nocite

%\printbibliography[keyword=pratique,heading=subbibliography,title={Théories littéraires dans les jeux vidéo}]
%\printbibliography[keyword=litteraire,heading=subbibliography,title={Narratologie et structuralisme}]

%\printbibliography[keyword=jeu,heading=subbibliography,title={\emph{Games studies}}]

\bibliographystyle{apalike}
%\bibliographystyle{plain}
\renewcommand{\bibname}{References}
\bibliography{Biblio.bib}

\cleardoublepage%


\printindex






\begin{abstract}

This mini project aims to develop an automated news classification system using machine learning. We built a 4-class classifier for Business, Sports, Technology, and World news articles using TF-IDF features and supervised learning.

Working with 144,522 articles from AG News and 20 Newsgroups datasets, we compared three algorithms: Logistic Regression, LinearSVC, and Random Forest. The best model (LinearSVC, C=0.1) achieved 90.70\% accuracy and F1-score of 0.9068, demonstrating effective text classification performance. \\

\keywordss{Text Classification, Machine Learning, TF-IDF, Support Vector Machines, News Categorization} \\

\vspace{1.5cm}
\centerline{\bf Résumé}
\vspace{0.6cm}

Ce mini projet vise à développer un système automatisé de classification d'articles de presse utilisant l'apprentissage automatique. Nous avons construit un classificateur à 4 classes pour les articles d'Affaires, Sports, Technologie et Monde en utilisant TF-IDF et l'apprentissage supervisé.

Travaillant avec 144,522 articles des corpus AG News et 20 Newsgroups, nous avons comparé trois algorithmes : Régression Logistique, LinearSVC et Forêts Aléatoires. Le meilleur modèle (LinearSVC, C=0.1) a atteint 90.70\% de précision et un score F1 de 0.9068, démontrant une performance efficace en classification de texte. \\

\keywords{Classification de Texte, Apprentissage Automatique, TF-IDF, Machines à Vecteurs de Support, Catégorisation d'Articles}
\end{abstract}




\end{document}
